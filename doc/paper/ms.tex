% mnras_template.tex
%
% LaTeX template for creating an MNRAS paper
%
% v3.0 released 14 May 2015
% (version numbers match those of mnras.cls)
%
% Copyright (C) Royal Astronomical Society 2015
% Authors:
% Keith T. Smith (Royal Astronomical Society)

% Change log
%
% v3.0 May 2015
%    Renamed to match the new package name
%    Version number matches mnras.cls
%    A few minor tweaks to wording
% v1.0 September 2013
%    Beta testing only - never publicly released
%    First version: a simple (ish) template for creating an MNRAS paper

%%%%%%%%%%%%%%%%%%%%%%%%%%%%%%%%%%%%%%%%%%%%%%%%%%
% Basic setup. Most papers should leave these options alone.
\documentclass[a4paper,fleqn,usenatbib]{mnras}

% MNRAS is set in Times font. If you don't have this installed (most LaTeX
% installations will be fine) or prefer the old Computer Modern fonts, comment
% out the following line
%\usepackage{newtxtext,newtxmath}
% Depending on your LaTeX fonts installation, you might get better results with one of these:
%\usepackage{mathptmx}
%\usepackage{txfonts}

% Use vector fonts, so it zooms properly in on-screen viewing software
% Don't change these lines unless you know what you are doing
\usepackage[T1]{fontenc}
\usepackage{ae,aecompl}


%%%%% AUTHORS - PLACE YOUR OWN PACKAGES HERE %%%%%

% Only include extra packages if you really need them. Common packages are:
\usepackage{graphicx}	% Including figure files
\usepackage{amsmath}	% Advanced maths commands
\usepackage{amssymb}	% Extra maths symbols

%%%%%%%%%%%%%%%%%%%%%%%%%%%%%%%%%%%%%%%%%%%%%%%%%%

%%%%% AUTHORS - PLACE YOUR OWN COMMANDS HERE %%%%%

% Please keep new commands to a minimum, and use \newcommand not \def to avoid
% overwriting existing commands. Example:
%\newcommand{\pcm}{\,cm$^{-2}$}	% per cm-squared

%%%%%%%%%%%%%%%%%%%%%%%%%%%%%%%%%%%%%%%%%%%%%%%%%%

%%%%%%%%%%%%%%%%%%% TITLE PAGE %%%%%%%%%%%%%%%%%%%

% Title of the paper, and the short title which is used in the headers.
% Keep the title short and informative.
\title[The Milky Way Dark Matter halo response to the Large Magellanic
Cloud]{The Milky Way's Dark Matter halo response to the Large Magellanic
cloud: Shape of the halo, time evolution and kinematics.}

% The list of authors, and the short list which is used in the headers.
% If you need two or more lines of authors, add an extra line using \newauthor
\author[N. Garavito-Camargo et al.]{
Nicol\'as Garavito-Camargo $^{1}$,\thanks{E-mail:
jngaravitoc@email.arizona.edu}
Gurtina Besla $^{1}$,
Kathryn V. Johnston $^{2}$,
Chervin Laporte $^{2}$,
Facundo A. G\'omez $^{3}$,
AP-W.
\\
% List of institutions
$^{1}$Steward Observatory, University of Arizona, 933 North Cherry
Avenue, Tucson, AZ 85721, USA.\\
$^{2}$Department of Astronomy, Columbia UNiversity, New York, NY
10027, USA.\\
$^{3}$Max-Planck-Institut fur Astrophysik, Karl-Scharzchild-Str. 1,
D-85748 Garching, Germany.
}

% These dates will be filled out by the publisher
\date{Accepted XXX. Received YYY; in original form ZZZ}

% Enter the current year, for the copyright statements etc.
\pubyear{2017}

% Don't change these lines
\begin{document}
\label{firstpage}
\pagerange{\pageref{firstpage}--\pageref{lastpage}}
\maketitle

% Abstract of the paper
\begin{abstract}
\end{abstract}

% Select between one and six entries from the list of approved keywords.
% Don't make up new ones.
\begin{keywords}
keyword1 -- keyword2 -- keyword3
\end{keywords}

%%%%%%%%%%%%%%%%%%%%%%%%%%%%%%%%%%%%%%%%%%%%%%%%%%

%%%%%%%%%%%%%%%%% BODY OF PAPER %%%%%%%%%%%%%%%%%%

\section{Introduction}

Dynamical tracers.
Mass estimates of the MW.
Formation history of the MW. Dynamics Time evolution. 

\cite{Bailin05}

\cite{Hernquist92}

\section{Numerical Methods:}

In order to study the response of the MW dark matter halo, we made the
following procedure; We run n-body simulations of the MW-LMC
system, this simulations reproduce the orbit of the LMC in a first
infall scenario following \verb+\citep{Besla07, etc..}+. We present 18
simulations corresponding to 3 different MW masses and 6 LMCs with
different masses. We then proceed to capture analytically the shape
and time evolution of the Dark Matter halo of the MW, for this aim we 
apply a basis function expansion at each snapshot of the simulation
and compute the amplitude of each multipole in an analogous
procedure as the one made in \verb+\citep{Lowing11}+. Once we have an
analytical description we proceed to compute orbits of test particles
at different galactocentric radius.


\subsection{MW and LMC models:}

We run tree different models for the MW with different masses. The
parameters of these models are listed in table \ref{tab:MWmodels}. 
The halo of the MW is represented by a Hernquist profile that enclosed
the same mass at the virial radius than a NFW profile for details of
this procedure see appendix of \verb+\citep{VdM12}+. The disk of the
MW is represented by an exponential disk whose scale high and scale
length are listed in table \ref{tab:MWmodels}. The Bulge of the MW is
modeled using a Hernquist profile. The corresponding rotation curves
of these models are shown in figure \ref{fig:MWmodels}.

\begin{table}
\begin{tabular}{c c c c c c c c c}
\hline
\hline
$M_{vir}$ & $R_{vir}$ & $c$ & $a$ & $M_{disk}$ & $r_a$ & $r_b$ & $M_{bulge}$
& $a_{bulge}$\\
\hline
1.2 & 279 & 15 & 40.85 & 5.78  & 2.9 & 0.638 & 1.4 & 0.7\\
 & & & & & & & & \\
 & & & & & & & & \\
\hline
\end{tabular}
\caption{\label{tab:MWmodels}}
\end{table}

The mass of the LMC is not well constrained, and it strongly depends
in where.

\begin{figure*}
 \centering
 \includegraphics[width=6.2in]{../../code/MW_models.pdf}
 \caption{ MW models}
 \label{fig:MWmodels}
\end{figure*}


\begin{table}
\begin{tabular}{c c c c c c c}
\hline
\hline
$M_{vir} [\times 10^{10}] M_{\odot}$ & $3$ & $5$ & $ 8$& $10$ & $18$ &
$25$ \\
$r_{s} [Kpc]$ & 3 & 6.4& 10.4& 12.7& 20 & 25.2\\
\hline
\hline
\end{tabular}
\caption{\label{tab:LMCmodels}}
\end{table}

\begin{figure*}
 \centering
 \includegraphics[width=6.2in]{../../code/LMC_models.pdf}
 \caption{ LMC models}
 \label{fig:LMCmodels}
\end{figure*}

We use the public available code \verb+GALIC+ \citep{Yurin14} in order
to initialize our MW and LMC models. 

\subsection{Milky Way and LMC N-body simulation}

\verb+GADGET-3+ \citep{Springel05}
softening length was computed following \citep{Power03} analysis.


The center of mass of the MW and the LMC where computes using a
shrinking sphere algorithm described in \citep{Power03}.


In order to find the initial conditions for the N-body simulations
we integrate the orbit of the LMC backwards in time from the observed
position and velocity derived from HST measurements by
\verb+\citep{Katyi}+. We use the dynamical friction term derived by 
Chandrasekhar equation \citep{Chandrasekhar44} where we take the
Coulomb Logarithm following \verb+\citep{Hashimoto03}+: 

\begin{equation}
Ln(\Lambda) = \alpha \left( \dfrac{b_{max}}{b_{min}} \right)
\end{equation}

Where $b_{max} = r$ is the galactocentric position of the LMC,
$b_{min}=1.4 * 3 kpc$ and $\alpha$ is a free parameter that 
softens the acceleration due to dynamical friction. We start with 
$\alpha=1$ and integrate the LMC up to the virial radius of the MW. 
Then we run a n-body simulation using these initial conditions, 
we iterate with lower values of $\alpha$ until we find a good agreement 
$2 \sigma$ (\textbf{change this value to the actual values}) with the
 observed positions and velocities of the LMC. These is done for all 
of our models, the orbits of the LMC are shown in figure $XX$ 
\textbf{Add figure with the orbits of the LMC}.

\subsection{SCF implementation}
\citep{Hernquist92}, \citep{Weinberg95}

\subsection{Orbits}



\section{Results:}


\section{Conclusions:}



\section*{Acknowledgements}
El Gato, 
This material is based upon work supported by the National Science
Foundation under Grant No. 1228509.

%%%%%%%%%%%%%%%%%%%%%%%%%%%%%%%%%%%%%%%%%%%%%%%%%%

%%%%%%%%%%%%%%%%%%%% REFERENCES %%%%%%%%%%%%%%%%%%

% The best way to enter references is to use BibTeX:

\bibliographystyle{mnras}
\bibliography{references} % if your bibtex file is called example.bib


% Alternatively you could enter them by hand, like this:
% This method is tedious and prone to error if you have lots of references

%%%%%%%%%%%%%%%%%%%%%%%%%%%%%%%%%%%%%%%%%%%%%%%%%%

%%%%%%%%%%%%%%%%% APPENDICES %%%%%%%%%%%%%%%%%%%%%

\appendix

\section{Triaxial halos SCF:}


%%%%%%%%%%%%%%%%%%%%%%%%%%%%%%%%%%%%%%%%%%%%%%%%%%


% Don't change these lines
\bsp	% typesetting comment
\label{lastpage}
\end{document}

% End of mnras_template.tex
