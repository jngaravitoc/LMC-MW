\label{sec:intro}
\section{Introduction}

In the current cosmological paradigm about 23\% of the energy and matter
content of the Universe is dark matter and $1\%$ is baryonic matter.
Understanding the nature of Dark matter is one of the most important challenges in
astrophysics. Astrophysical constraints are useful to test the predictions
of the dark matter nature. (modify last sentence)\\

In the context of the $\LambdaCDM$ where structure grows hierarchically, baryonic matter
accumulates within dark matter halos, allowing galaxies to form. Therefore
Galaxy formation processes are strongly related with the evolution of dark
matter halos. In particular, the morphology, kinematics and dynamics of galaxies
disks and satellite remnants of hierarchical processes
are useful observables that help to constrain halos properties such as
the total mass, density and shape.\\

Crucial theoretical efforts have been made to model the properties of dark
matter halos. Hernquist, Plummer, Dehnen, derived analytical expressions for
the density profiles of spherical dark matter halos. Using cosmological
simulations Navarro et al found an analytical expression for spherical
dark matter profiles in a cold dark matter universe. More recent studies have
suggested that halos in a $\Lambda CDM$ Universe are triaxial (Jung\& Suto 01).\\


Satellite galaxies and globular clusters that orbit around the host
galaxies can trace the mass and shape of the halo if the position, line of
sight velocities and tangential velocities were known. Such observations
have been made for several objects around the Milky Way allowing different 
research groups to constrain the mass of the Milky Way but there is not a consistent
agreement yet. The best example is the Sagittarius stream.\\

Due to the dynamical friction satellite galaxies and globular clusters would decay into 
the halo of the MW. During this process the satellite would be distroyed leaving 
tidal streams (toomretoomre72), these streams where expected to be observed 
in the Milky Way (lblb95, jhonston96). 
The first satellite galaxy observed inside the Milky Way halo was the Sagittarius dwarf 
galaxy (ibata94) and the streams were observed (ivezic00, yanny00, ibata01b). 
A full observation of the stream was made by (majewsky03), the stream  
have a width ($>20°$) (belokurov06) and it warps
for more than $360°$(majewsky03, piladiaz13, belokurov14) around the milky way, 
the velocity space is ($>20km/s$) (koposov13) which suggest that
the Sgr stream is dynamically hot.\\

\vspace{16cm}

With these observations several groups have attempt to reproduce the stream and found 
different halo shapes, (Ibata et al 01 (Spherical)
,Helmi 04 (Prolate), Johnston et al 05 (Oblate halo), Law et al 05,Fellhauer et al 06 (Spherical)
, Martinez Delgado et al 07 (Oblate), Law, Majewski, Johnston 09, Law \& Majewski (Triaxial),
Vera-Ciro 2013(Oblate and triaxial). From these meodels the Law \& Majewsky model has been 
the more succesfull reproducing the observed positions and velocites of the stram. Nevertheless
its configuration is unusual and would not be favorable for the formation of the Milky Way Disk (Debattista13).

These disagreemnts motivates the use of other streams,
recently Pearson14 used the Palomar 5 stream to constrain the dark
matter halo of the MW. The stream is 22 degrees
long and with a width of 0.7 degrees. The globular cluster is currently at the
apocenter of the orbit at 23.6kpc from the Sun (Dotter et al 2011). Interestingly
the potential that reproduce the morphology of Palomar 5 is spherical, in
disagreement with the models that reproduce the Sagittarius stream.

New techniques are being developed with the aim of constraining the halo with multiple
streams at different distances at the same time (Bonaca14). These techniques 
prevent the underestimate or overestimate of the halo mass.

However all of the studies that constrain the dark matter halo using streams
assume a static dark matter halo, (Buist \& Helmi15) claim that if the dark matter
halo is changing in time misalignments on the orbit of long streams could appear.

The Milky Way dark matter is not isolated in fact it is interacting with a 
massive satellite ~$1\times10^{11}M_{\odot}$, the Large Magellanic Cloud (LMC)
which is in it's first passage about the MW and is currently at the pericenter
at 50Kpc from the galactic center. Such a massive satellite inside the virial 
radius ($R_{vir}\sim 261kpc$) of the MW should be distorting the MW dark matter
halo.

A detail study of the influence of the LMC in the MW dark matter halo
shape is needed in order to prevent biases in the streams models. 
The aim of this work is study these influence for this aim 
we model the MW as an spherical halo that is distorted by the LMC. We integrate 
analytically the orbit of the LMC until the MW halo virial radius of $261Kpc$, 
from these initial conditions we used N-body simulations to study the distortion 
of the LMC in a first passage scenario.   
  

