\label{sec:intro}
\section{Introduction}

In the current cosmological paradigm about 23\% of the energy and matter
content of the Universe is dark matter and $1\%$ is baryonic matter.
Understanding the nature of dark matter is one of the most important challenges in
astrophysics. Astrophysical constraints are useful to test the predictions
of the dark matter nature. %(modify last sentence)

In the context of the $\Lambda CDM$, where structure grows hierarchically, baryonic matter
accumulates within dark matter halos, allowing galaxies to form.
Therefore, galaxy formation processes are strongly related with the evolution of dark
matter halos. In particular, the morphology, kinematics and dynamics of galaxies
disks and satellite remnants of hierarchical processes
such as streams are useful observables that help to constrain halos'
properties like their total mass, density and shape.

Crucial theoretical efforts have been made to model the properties of dark
matter halos. \citep{Hernquist90}, Plummer, Dehnen, derived analytical expressions for
the density profiles of spherical dark matter halos. Using cosmological
simulations \citep{Navarro96, Navarro97} found an analytical expression for spherical
dark matter profiles in a cold dark matter universe. More recent studies have
suggested that halos in a $\Lambda CDM$ Universe are triaxial (Jung\& Suto 01).

Satellite galaxies and globular clusters that orbit around the host
galaxies can trace the mass and shape of the halo if the position, line of
sight velocities and tangential velocities are known. Such observations
have been made for several objects around the Milky Way, allowing different
research groups to constrain the mass of the Milky Way, nevertheless, there is not a consistent
agreement yet. %(Put an example)

Due to the dynamical friction satellite galaxies and globular clusters
will decay over time. During this process the satellite will be
destroyed leaving tidal streams as tracers of this process
\citep{Toomre72, LyndenBell95, Johnston96}. These streams are observed in the Milky Way
, The first disrupting satellite galaxy observed inside the Milky Way halo was the Sagittarius dwarf
galaxy \citep{Ibata94}. Sections of streams associated with Sagittarius were
observed in (ivezic00, yanny00, ibata01b).
These streams segments were revealed to be connected, forming a stream
spanning for more than $360°$ (majewsky03, piladiaz13, belokurov14), the stream have a width ($>20°$) (belokurov06) and it warps
for more than $360°$. The velocity space is ($>20km/s$) (koposov13)
which suggests that
the Sgr stream is dynamically hot.

\begin{table}
\begin{center}
\begin{tabular}{c c c}
\hline
\hline
Halo Shape & Author & Parameters \\
\hline
Spherical & Ibata et al 01 & \\
Spherical & Law05 & \\
Spherical & Fellhauer06 & \\
Prolate & Helmi 04 & \\
Oblate & Johnston05 & \\
Oblate & MartinezDelgado07 & \\
Triaxial & LMJ09 & \\
Triaxial & LM & \\
Oblate-Triaxial & Vera-Ciro13 & \\
\hline
\hline
\end{tabular}
\caption{Halo Shape models derived from streams.\label{tab:models}}
\end{center}
\end{table}


With these observations several groups have attempted to reproduce the
stream with N-body models \ref{tab:models}, founding different halo
shapes. From these models, the \citep{Law10} model has been
the more successful by reproducing the observed positions and velocities
of the stream. Nevertheless, it's halo configuration is unusual
because the major axis is perpendicular to the plane of the disk; this
configuration is not favorable for the formation of the
Milky Way Disk \citep{Debattista13}.

These disagreements motivate the use of other streams to map the halo,
recently \citep{Pearson15} used the Palomar 5 (Pal5) stream to constrain the dark
matter halo of the Milky Way. The stream is $22\degree$
long and with a width of $0.7 \degree$. The globular cluster is currently at the
apocenter of the orbit at 23.6kpc from the Sun (Dotter et al 2011). Interestingly
the potential that reproduce the morphology of Pal5 is spherical.
Recent observations of Pal5 \citep{Fritz15} also suggest a spherical
halo with $V_0 = 220 km/s$ and $V_{20kpc} = 218 km/s$. These results
are in disagreement with the \citep{Law10} model that reproduce the Sagittarius stream
and also such a halo shape is uncommon in the $\Lambda$CDM
paradigm. %\verb+ponercita+.

New techniques are being developed with the aim of constraining the halo with multiple
streams at different distances at the same time (Bonaca14). These techniques
prevent the underestimation or overestimation of the halo mass.
%\textbf{why?}

However, all of the studies that constrain the dark matter halo using streams
assume a static dark matter halo. \citep{Buist15} claim that if the dark matter
halo is changing in time misalignments on the orbit of long streams could appear.
%\textbf{explain}

The Milky Way is not isolated, in fact it is interacting with a
massive satellite ~$1\times10^{11}M_{\odot}$ (\textbf{where this came
from}), the Large Magellanic Cloud (LMC),
which is in the first passage about the MW and is currently at the pericenter
of the orbit at 50Kpc from the Galactic center. Such a massive satellite inside the virial
radius ($R_{vir}\sim 261kpc$) of the Milky Way should be distorting
the it's dark matter
halo. %\textbf{Why? Compre MW rot cum, mass enclosed within 50kpc & Mass
%enclosed of LMC. Create a simple model Gomez2015, orbital barycenter
%changes. 1:4 merger}

Due to the LMC, the dark matter halo of the Milky Way might be
distorted and evolving in time. Consequently, a detailed study of the
influence of the LMC on the MW dark matter halo shape is needed. Two
main questions are addressed in this work: What are the evolving shape of the inner
(<20kpc) and outer halo (>20kpc) and what are the torques applied
to streams in the inner halo.

We use N-body simulations, which reproduce the orbital history of the
Large Magellanic Cloud around the Milky Way. In these simulations the Milky Way is modeled
using an spherical halo that is distorted by the Large Magellanic
Cloud.
In \S\ref{sec:models} we described the models used for the Large
Magellanic Cloud and
the Milky Way. In \S\ref{sec:IC} we explained how we derived our initial
conditions that reproduce the orbital history of the Large Magellanic
Cloud. In \S\ref{sec:shapes} we explained the methods used to study
the shape of the halos. In \S\ref{sec:results} we present our
results.
