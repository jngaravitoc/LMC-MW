\label{sec:intro}
\section{Introduction}

In the current cosmological paradigm about 23\% of the energy and matter
content of the Universe is dark matter and $1\%$ is baryonic matter.
Understanding the nature of Dark matter is one of the most important challenges in
astrophysics. Astrophysical constraints are useful to test the predictions
of the dark matter nature. (modify last sentence)

In the context of the $\LambdaCDM$ where structure grows hierarchically, baryonic matter
accumulates within dark matter halos, allowing galaxies to form. Therefore
Galaxy formation processes are strongly related with the evolution of dark
matter halos. In particular, the morphology, kinematics and dynamics of galaxies
disks and satellite remnants of hierarchical processes
are useful observables that help to constrain halos properties such as
the total mass, density and shape.

Crucial theoretical efforts have been made to model the properties of dark
matter halos. Hernquist, Plummer, Dehnen, derived analytical expressions for
the density profiles of spherical dark matter halos. Using cosmological
simulations Navarro et al found an analytical expression for spherical
dark matter profiles in a cold dark matter universe. More recent studies have
suggested that halos in a $\Lambda CDM$ Universe are triaxial (Jung\& Suto 01).

Satellite galaxies and globular clusters that orbit around the host
galaxies can trace the mass and shape of the halo if the position, line of
sight velocities and tangential velocities were known. Such observations
have been made for several objects around the Milky Way allowing different 
research groups to constrain the mass of the Milky Way but there is not a consistent
agreement yet. The best example is the Sagittarius stream.

Due to the dynamical friction satellite galaxies and globular clusters
would decay over time. During this process the satellite would be
destroyed leaving tidal streams as tracers of this process
(toomretoomre72). These streams are observed in the Milky Way (lblb95, jhonston96).
The first disrupting satellite galaxy observed inside the Milky Way halo was the Sagittarius dwarf 
galaxy (ibata94). Sections of streams associated with Sag were
observed in XXX (ivezic00, yanny00, ibata01b).
These streams segments were revealed to be connected, forming a stream
spanning $360°$ (majewsky03), the stream have a width ($>20°$) (belokurov06) and it warps
for more than $360°$(majewsky03, piladiaz13, belokurov14) around the
Milky Way, the velocity space is ($>20km/s$) (koposov13) which suggest that
the Sgr stream is dynamically hot.

%\vspace{16cm}

With these observations several groups have attempted to reproduce the
stream with N-body models. Founding different halo shapes,

\begin{table}
\begin{center}
\begin{tabular}{c c c}
\hline
\hline
Halo Shape & Author & Parameters \\
\hline
Spherical & Ibata et al 01 & \\
Spherical & Law05 & \\
Spherical & Fellhauer06 & \\
Prolate & Helmi 04 & \\
Oblate & Johnston05 & \\
Oblate & MartinezDelgado07 & \\
Triaxial & LMJ09 & \\
Triaxial & LM & \\
Oblate-Triaxial & Vera-Ciro13 & \\
\hline
\hline
\end{tabular}
\caption{Halo Shape models derived from streams.}
\end{center}
\end{table}

From these models the Law \& Majewsky model has been 
the more succesfull reproducing the observed positions and velocities
of the stream. Nevertheless it's halo configuration is unusual and would 
not be favorable for the formation of the Milky Way Disk (Debattista13).

These disagreements motivate the use of other streams to map the halo,
recently Pearson14 used the Palomar 5 stream to constrain the dark
matter halo of the MW. The stream is 22 degrees
long and with a width of 0.7 degrees. The globular cluster is currently at the
apocenter of the orbit at 23.6kpc from the Sun (Dotter et al 2011). Interestingly
the potential that reproduce the morphology of Palomar 5 is spherical.
Recent observations of Pal5 \citep{Fritz15} also suggest a spherical
halo with $V_0 = 220 km/s$ and $V_{20kpc} = 218 km/s$. These results
are in disagreement with the \citep{LawMajewsky} model that reproduce the Sagittarius stream
and a such a halo shape would be uncommon in the $\Lambda$CDM
paradigm \verb+ponercita+.

New techniques are being developed with the aim of constraining the halo with multiple
streams at different distances at the same time (Bonaca14). These techniques 
prevent the underestimation or overestimation of the halo mass.
\textbf{why?}

However all of the studies that constrain the dark matter halo using streams
assume a static dark matter halo, (Buist \& Helmi15) claim that if the dark matter
halo is changing in time misalignments on the orbit of long streams could appear.
\textbf{explain}

The Milky Way is not isolated, in fact it is interacting with a 
massive satellite ~$1\times10^{11}M_{\odot}$, the Large Magellanic Cloud (LMC)
which is in the first passage about the MW and is currently at the pericenter
of the orbit at 50Kpc from the Galactic center. Such a massive satellite inside the virial 
radius ($R_{vir}\sim 261kpc$) of the MW should be distorting the MW dark matter
halo. \textbf{Why? Compre MW rot cum, mass enclosed within 50kpc & Mass
enclosed of LMC. Create a simple model Gomez2015, orbital barycenter
changes. 1:4 merger}


Due to the LMC the dark matter halo of the Milky Way might be
distorted and evolving in time. Consequently a detailed study of the
influence of the LMC on the MW dark matter halo shape is needed. Two
main questions are addressed in this work; What are the evolving shape of the inner
(<20kpc) and outer halo (>20kpc). Study the torques applied to streams in the inner halo.

We use N-body simulations which reproduce the orbital history of the
LMC. We model the MW as an spherical halo that is distorted by the LMC.
In \S\ref{sec:models} we described the models used for the LMC and
the MW. In \S\ref{sec:IC} we explained how we derived our initial
conditions. In \S\ref{sec:shapes} we explained the methods to study
the shape of the halos. In \S\ref{sec:results} we present our
results.
