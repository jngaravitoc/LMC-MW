%\chapter{models}\label{sec:shapes}
\section{Appendix}\label{sec:Appendix}

\subsection{Milky Way model derivation}

The input parameters are resumed in table:

\begin{table}
\begin{center}
\begin{tabular}{c c}
$V_{200}$ & $175 km/s$ \\
c200 & 9.6 \\
Mdisk & 0.0524 \\
Mbulge & 0.0082 \\
\end{tabular}
\end{center}
\end{table}

\begin{equation}\label{eq:c200cvir}
\dfrac{c_{200}}{c_{vir}} = \left( \dfrac{f(c_{200})}{qf(c_{vir})}\right)^{1/3}
\end{equation}

\begin{equation}
\dfrac{a}{r_s} = \left(\dfrac{1}{\dfrac{1}{\sqrt{2f(\tilde{x})} -
\dfrac{1}{\tilde{x}}} \right)
\end{equation}

\begin{equation}
\dfrac{M_H}{M_{vir}} = \dfrac{\left(\dfrac{a}{r_s}
\right)^2}{2f(c_{vir})}
\end{equation}

The corresponding Mass, radius at $r_{200}$ is:

$M_{200} = 1.77 \times 10^{12}M_{\odot}$ This is the Hernquist halo
mass at $200\rho_c$.
$R_{200} = 250 kpc$

The scale length of the NFW profile at $r_{200}$ is:

$r_{s, 200} = 26.04 kpc$ This is the scale length of the NFW profile
at $200\rho_c$.

The concentration $c_{200}$ is for the NFW profile at $200\rho_c$,
using eq.\ref{eq:c200cvir} $c_{vir}=12.86$.

With $c_{200}$ can compute $a_{200}/r_{s,200} = 2.07, 1.7 kpc$
respectively for the exact relation in VDM and for the approximated
solution of Springel.

$a_{200} = 54, 44 kpc$.

Now we have to compute the NFW profile that encloses the same mass at
$r_{200}$, this is:







