\chapter{models}\label{sec:shapes}
\section{Halo Shapes}\label{sec:shapes}

Measuring halo shapes:

\subsection{Inertia Tensor Method}

The first step in a better description of dark matter halos shapes
beyond the spherical approximation is assuming that halos are
ellipsoids. Ellipsoids can be classified as oblate, prolate or
triaxial. The ellipsoids equation is:

%\begin{equation}
%\dfrac{x^2}{a^2} + \dfrac{y^2}{b^2} + \dfrac{z^2}{c^2} = R^2
%\end{equation}

Where $a, b, c$ are the length of the principal axis of the
ellipsoid and the relation $a \geq b \geq c$.

\begin{cases}
Oblate if $a>b \sim c$\\
Prolate if $ a \sim b>c$\\
Triaxial if $a>b>c$\\
\end{cases}

Dark Matter halos can be characterized into ellipsoids by computing
the Inertia Tensor method defined as:

\begin{equation}
I_{i,j} = \sum_n x_{i,n} x_{j,n}
\end{equation}

The eigenvalues ($\lambda_a, \lambda_b, \lambda_c$) and eigenvectors
of the Inertia tensor provide the principal axis length and directions
respectively.

\subsection{Expansions method}




